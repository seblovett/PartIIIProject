%% ----------------------------------------------------------------
%% Research.tex
%% ---------------------------------------------------------------- 
\chapter{Research} \label{Chapter:Research}
The research done for this project is split down into three sections:
\begin{enumerate}
\item Hardware
\item Software, broken down into:
\begin{enumerate}
\item Firmware
\item Algorithms
\end{enumerate}
\end{enumerate}

Hardware and Firmware research will be discussed in this section. Vision algorithms are looked at in detail in chapter \ref{Chapter:InvestigationVision}.
\section{Hardware Research}
%\inote{Talk about why I chose to develop with AVRs, comparison with other uControllers. Why I used the OV7670 Camera etc etc}
\subsection{Microcontrollers}
The robot is to be designed with a budget of \pound 80 (not including P.C.B.). The choice of microcontroller will be an important one as a compromise between cost, power and usability must be made. There are two main brands of microcontrollers present in the consumer market: ARM and Atmel AVRs.\\ %and PICs. 
%\inote{Do more research into ARM stuff}
ARM is an architecture which is developed by ARM Holdings. ARM devices come in a many varieties, ARM9, ARM7, Strong ARM, ARM Cortex etc. While ARM Holdings do not fabricate and sell the devices themselves, many companies, such as Texas Instruments, use the architecture and fabricate their own devices. For this comparison, the Stellaris by Texas Instruments will be looked at. ARM cores tend to be 32-bit and have a high clock speed. They are a RISC Harvard architecture. ARM microcontrollers have onboard support for SPI, \itc, PWN, ADCs and have Flash, SRAM and EEPROM memory. 

Atmel have a variety of products in the microcontroller market. They range from 8-bit, low clock speed packages for the hobbyist (ATMega, ATTiny), an improved 8-bit variant, XMega, and a 32-bit design, AT32UC3C. XMegas and AVR32s tend to have higher clock speeds than the ATMegas, around 30MHz for Xmegas and 60MHz for AVR32s. The AVR core is also a Harvard RISC architecture, but mainly 8-bit. Atmel devices often have on board peripherals such as \itc (called TWI on AVRs), SPI, ADCs as well as a number of different memories: Flash, EEPROM and SRAM. An AT32UC3C0512C, ATXmegaA3BU and ATMega644P will be compared in this section. 


Table \ref{tab:uCComp} shows a brief summary of some common ARM and AVR microcontrollers. The Stellaris offers a log of  power with the largest DMIPS performance. However, due to the necessity of floating point operations, the AT32 clearly has a distinct advantage by having a built in floating point unit. The XMega and ATMega do not offer enough power and are restricted with a small amount of SRAM and Flash. All devices looked at use 3.3V supply and have basic communication protocols (SPI, \itc). Overall, the AT32UC3C0512C is the best choice with a high throughput, a floating point unit and a vast amount of GPIO and communications. There is no EEPROM which may be desired to have, but these can be added onto and SPI or \itc bus. 
%\inote{Much point wasting words on PIC?}
%PICs are useless...

\begin{table}
\centering
\begin{tabular}{|p{3.5cm}|p{2.4cm}|p{2.9cm}|p{2.3cm}|p{2.4cm}|}\hline
						& 	ARM Stellaris		&	AT32UC3C0512C 		&	XMegaA3BU	&	ATMega644P 	\\	\hline
Clock Speed	(MHz)		&	80					&	33 or 66			&	32			&	12			\\
DMIPS					&	100					&	91					&	-			&	20 MIPS		\\
Package					&	100 LQFP or 108 BGA	&	64, 100, 144TQFP	&	64 QFP or QFN & 40 DIP, 44 TQFP, 44 QFN \\
Cost of 1 unit(\pounds)	&	10.30				& 15.39 &	6.65	 & 6.86\\
Flash Size(kB)			&	256					&	512					&	256			&	64 \\
SRAM Size (kB)			&	32					&	64					&	16			&	4	\\
EEPROM Size(kB)			&	2					&	None internal		&	4			&	2 	\\
GPIO					&	64					& 	45, 81 or 123		&	47			& 	32	\\
Operating Voltage (V)	&	3.3					& 	5	or 3.3			& 	1.6- 3.6\footnote{Clock frequency for voltages 1.6 - 2.7 is 12MHz.}		& 	2.7-5.5	\\
Communication Interfaces &	SPI, \itc, SSI, MAC, CAN, EPI, USB, USART, I2S	& SPI, TWI, EBI, USB, Ethernet, CAN, USART, I2S	&	USART, TWI, USB, SPI 		&	SPI, TWI, USART \\
Floating Point			&	None				&	Built in FPU		&	None		&	None		\\
ADCs					&	16					&	16					&	16			&	8			\\
Timers					&	4					&	3 16-bit			& 7 16-bit, 8 8-bit & 2 8-bit, 1 16-bit \\
\hline
\end{tabular}
\caption{Comparison Table of some common microcontrollers. Data of microcontroller taken from }
\label{tab:uCComp}
\end{table}

%\subsection{Development Boards}


\section{Firmware}
\subsection{Camera}

The camera used is the OV7670 camera by OmniVision. Steve Gunn provided source code for use on the Il Matto development board which uses an ATMega644P and has an onboard SD Card reader. It is supplied on a small breakout board with a FIFO buffer. The camera operation is discussed in section \ref{Section:Camera}. Many implementations of firmware for this camera exist

\subsection{Atmel Software Framework}

Atmel offer a Software Framework which contains basic code and device drivers for many of their XMega and AT32 devices \cite{Atmel:ASF}. There are also many AVR Application notes which provide explanations and example code for things like TWI, SPI and timers. These application notes are aimed at older devices like the ATTiny and ATMega and are generally written for IAR Embedded Workbench compiler, as opposed to the AVRGCC compiler used within Atmel Studio. 

%\section{Image Algorithms}
%
%\subsection{Comparison Algorithms} 
%
%\subsection{Detection Algorithms}
%It will be necessary to be able to work out what in the image are objects. For this, a number of detection algorithms can be used:
%\begin{enumerate}
%\item Corner Detection
%\item Edge Detection
%\end{enumerate}
%
%\subsection{Corner Detection}
%
%A common edge detection algorithm is the Harris Corner Detector \cite{Nixon:HarrisAlgo}. This works by placing a window over the image and measuring the intensity. If the intensity changes in both vertical and horizontal directions, there is a corner. If the window is on an edge, the intensity will only change in the direction perpendicular to the edge. This method will be useful to detect an object. However, this may not be able to detect a tall, vertical object, such as the edge of a wall.
%
%\subsection{Edge Detection}
%There are many edge detection algorithms. The 


